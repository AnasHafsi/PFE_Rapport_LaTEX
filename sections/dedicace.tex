\newenvironment{dedication}
     {\vspace{6ex}\begin{quotation}\begin{center}\begin{em}}
     {\par\end{em}\end{center}\end{quotation}}

     \begin{dedication}
Je dedie mi trabajo a 
Whoever has seen deeply into the world has doubtless divined what wisdom there is in the fact that men are superficial. It is their preservative instinct which teaches them to be flighty, lightsome, and false. Here and there one finds a passionate and exaggerated adoration of "pure forms" in philosophers as well as in artists: it is not to be doubted that whoever has NEED of the cult of the superficial to that extent, has at one time or another made an unlucky dive BENEATH it. Perhaps there is even an order of rank with respect to those burnt children, the born artists who find the enjoyment of life only in trying to FALSIFY its image (as if taking wearisome revenge on it), one might guess to what degree life has disgusted them, by the extent to which they wish to see its image falsified, attenuated, ultrified, and deified,—one might reckon the homines religiosi among the artists, as their HIGHEST rank. It is the profound, suspicious fear of an incurable pessimism which compels whole centuries to fasten their teeth into a religious interpretation of existence: the fear of the instinct which divines that truth might be attained TOO soon, before man has become strong enough, hard enough, artist enough.... Piety, the "Life in God," regarded in this light, would appear as the most elaborate and ultimate product of the FEAR of truth, as artist-adoration and artist-intoxication in presence of the most logical of all falsifications, as the will to the inversion of truth, to untruth at any price. Perhaps there has hitherto been no more effective means of beautifying man than piety, by means of it man can become so artful, so superficial, so iridescent, and so good, that his appearance no longer offends.
\end{dedication}