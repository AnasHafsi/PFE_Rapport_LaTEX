Une étape cruciale dans le cycle de développement des systèmes software adopté par toute société respectant
les normes et standards du domaine automobile est l’ingénierie des exigences (RE). Celle-ci fut créée comme
sous-domaine de l’ingénierie logicielle avec la tâche de développer des modèles, des techniques et des outils qui
permettent de hiérarchiser les exigences d’un projet, de détecter les incohérences entre elles et d’assurer leur
traçabilité. Elle permet donc d’éviter des frais onéreux engendrés par une maintenance couteuse des systèmes
logiciels, due à des erreurs au niveau de la définition des exigences, qui pourrait même entrainer à un rejet total,
au pire des cas.
L’ingénierie des exigences s’articule principalement sur deux phases : la définition des exigences (RD) et la
gestion des exigences (RM). Le développement des exigences permet d’obtenir un ensemble de spécifications
convenues. Cependant, les changements de ces spécifications sont souvent inéluctables, et les facteurs sont multiples : nouvelles priorités d’affaires, découverte de nouvelles exigences, etc. La gestion des exigence, quant à
elle, surveille les changements et garantit que les exigences sont modifiées de manière contrôlée, en maintenant
la traçabilité tout au long de la réalisation du projet.
C’est dans ce cadre que Lear CORPORATION a adopté l’outil Rational Doors, et a développé l’outil ReqTool, qui exploite les données de DOORS ainsi que différentes sources additionnelles (SQA, ReqDB ...) pour
générer des rapports et des matrices de traçabilités. Néanmoins, ces derniers restent peu pratiques due à leur
format apathique et non intuitif. De plus, bien qu’il soit nécessaire de les générer plusieurs fois par jour, le
temps de génération est si important qu’on ne peut se permettre de le faire que deux à trois fois au plus, par
jour. Une révision de l’outil s’impose donc, afin de l’optimiser.
Mots-clés : Gestion des exigences, Cycle en V, Rational DOORS, DXL, ReqTool, Modélisation des données,
Spring Boot, ReactJs, JEE, Micro-services.