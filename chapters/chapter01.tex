% Chapter Template

\chapter{Organisme d’accueil} % Main chapter title

\label{Chapter01} % Change X to a consecutive number; for referencing this chapter elsewhere, use \ref{ChapterX}

\par
Avant de commencer par l’analyse du projet il est essentiel de commencer par une présentation générale des entités d’accueil, ceci en introduisant l'historique ainsi que les produits. Et donc par conséquent, Ce chapitre est une présentation sommaire du contexte de notre projet de fin d’études. Il comporte deux grandes parties : dans la première, nous allons présenter l’entreprise d’accueil Finaxys Paris, puis nous allons présenter Amundi Asset Manager, le client de Finaxys chez lequel j'ai pu réaliser le projet.
Chaque partie est subdivisée en plusieurs sous-partie : nous exposeront l'historique de chaque entreprise, nous allons dévoiler les chiffres clés démontrant son succès économique. Nous allons ensuite expliquer brièvement l'organisation fonctionnelle ainsi que les différentes équipes de IT. sa démarche qualité pour la gestion de ses projets. Dans la seconde partie, nous présenterons notre client et développerons le cadre général du projet CDL; Carnet de Liaison en présentant brièvement ses modules, notamment les modules Adhérent et Suivi qui font l’objet de notre PFE. Dans la dernière partie, nous allons révéler la conduite de projet pour laquelle nous avons opté afin de mener efficacement notre projet.


\section{Finaxys}
\par
Finaxys est une société de consultants spécialisée en IT finance, banque et assurance. Elle permet aux acteurs de ce marché de bénéficier d’experts pour l’analyse, le cadrage et la définition des projets IT, comme pour leur déploiement et implémentation.
\subsection{Historique}

Nunc posuere quam at lectus tristique eu ultrices augue venenatis. Vestibulum ante ipsum primis in faucibus orci luctus et ultrices posuere cubilia Curae; Aliquam erat volutpat. Vivamus sodales tortor eget quam adipiscing in vulputate ante ullamcorper. Sed eros ante, lacinia et sollicitudin et, aliquam sit amet augue. In hac habitasse platea dictumst.

%-----------------------------------
%	SUBSECTION 2
%-----------------------------------

\subsection{Domaine d'activité}
Morbi rutrum odio eget arcu adipiscing sodales. Aenean et purus a est pulvinar pellentesque. Cras in elit neque, quis varius elit. Phasellus fringilla, nibh eu tempus venenatis, dolor elit posuere quam, quis adipiscing urna leo nec orci. Sed nec nulla auctor odio aliquet consequat. Ut nec nulla in ante ullamcorper aliquam at sed dolor. Phasellus fermentum magna in augue gravida cursus. Cras sed pretium lorem. Pellentesque eget ornare odio. Proin accumsan, massa viverra cursus pharetra, ipsum nisi lobortis velit, a malesuada dolor lorem eu neque.

\subsection{Fiche Technique}
\begin{table}[htp]
    \begin{center}
        \begin{tabular}{ | l | l |}
            \hline
            Day & Min Temp  \\ \hline
            Monday & 11C \\ \hline
            Tuesday & 9C  \\ \hline
            Wednesday & 10C  \\
            \hline
            \end{tabular}
            \caption{Fiche Technique Finaxys}
    \end{center}
\end{table}


\section{Amundi Asset Management}

Sed ullamcorper quam eu nisl interdum at interdum enim egestas. Aliquam placerat justo sed lectus lobortis ut porta nisl porttitor. Vestibulum mi dolor, lacinia molestie gravida at, tempus vitae ligula. Donec eget quam sapien, in viverra eros. Donec pellentesque justo a massa fringilla non vestibulum metus vestibulum. Vestibulum in orci quis felis tempor lacinia. Vivamus ornare ultrices facilisis. Ut hendrerit volutpat vulputate. Morbi condimentum venenatis augue, id porta ipsum vulputate in. Curabitur luctus tempus justo. Vestibulum risus lectus, adipiscing nec condimentum quis, condimentum nec nisl. Aliquam dictum sagittis velit sed iaculis. Morbi tristique augue sit amet nulla pulvinar id facilisis ligula mollis. Nam elit libero, tincidunt ut aliquam at, molestie in quam. Aenean rhoncus vehicula hendrerit.
\subsection{Un Asset Manager}

\par Par abus de langage le terme « asset manager » est ici utilisé pour décrire le marché des biens immobiliers ; les Anglo-saxons associent le terme asset management à une gestion beaucoup plus étendue d'actifs, comprenant entre autres les obligations, les actions, les produits dérivés, les matières premières...
\par L’asset manager immobilier est un terme anglo-saxon, apparu en France dans les années 1990, qui définit le métier de gestionnaire d'actifs immobiliers d'entreprise comme un ensemble de services aux propriétaires, de l'acquisition du bien à sa cession.
\par L’asset manager immobilier est le responsable de la gestion d'un portefeuille d'actifs immobiliers pour le compte de tiers.
\par Il est le garant de la rentabilité des biens immobiliers attendue par les investisseurs, assurée par sa connaissance polyvalente des marchés de l'immobilier d'entreprise et d'habitation, sa maîtrise des baux commerciaux et du droit immobilier, ses compétences financières tant au plan de la modélisation de flux de trésorerie et scénarios immobiliers qu'il doit optimiser que du reporting aux investisseurs, sa capacité à penser et suivre des plans de travaux sur les immeubles, de la restructuration lourde au développement et son talent de négociateur tant pour la commercialisation des actifs que pour ses relations avec les différents intervenants et prestataires extérieurs.
\par Ce métier est apparu en France vers la fin des années 1990, après la crise immobilière sans précédent qu'avait connue le pays, à la suite de la spéculation de nombreux acteurs financiers anglo-saxons qui avaient su saisir les opportunités durant cette période et avaient en particulier apporté des techniques d'investissement et de gestion encore inconnues en France.
\par Une pyramide des métiers s'est créée et régit aujourd'hui le secteur de l'immobilier d'entreprise : l’investment manager immobilier (responsable d'investissements), l’asset manager immobilier (gestionnaire d'actifs immobilier représentant du propriétaire), le fund manager immobilier (responsable de fonds d'investissement immobilier), le portfolio manager immobilier (responsable des relations avec les clients et du reporting financier), le property manager (gestionnaire locatif et technique), le facility manager (responsable des services aux utilisateurs de l'immeuble).
\par Selon les sociétés, il est possible de trouver des définitions différentes de ces métiers ; l’asset manager immobilier peut ainsi, en sus de sa casquette de gestionnaire, être responsable de l'investissement en amont et de la vente de l'actif à un tiers.
\par L’asset manager immobilier exerce son métier sur des types d'actifs variés : bureaux, centres commerciaux, commerces de pied d'immeuble, logistique, entrepôts, habitation, hôtels… Il se doit de maîtriser les particularités afférentes à chacun de ces secteurs d'activités, par exemple : le marché de l'emploi et du tertiaire pour les bureaux, l'actualité des enseignes et la consommation des ménages pour le commerce, la réglementation des installations classées pour la logistique, etc.

\subsection{Historique}

Nunc posuere quam at lectus tristique eu ultrices augue venenatis. Vestibulum ante ipsum primis in faucibus orci luctus et ultrices posuere cubilia Curae; Aliquam erat volutpat. Vivamus sodales tortor eget quam adipiscing in vulputate ante ullamcorper. Sed eros ante, lacinia et sollicitudin et, aliquam sit amet augue. In hac habitasse platea dictumst.

%-----------------------------------
%	SUBSECTION 2
%-----------------------------------

\subsection{Domaine d'activité}
Morbi rutrum odio eget arcu adipiscing sodales. Aenean et purus a est pulvinar pellentesque. Cras in elit neque, quis varius elit. Phasellus fringilla, nibh eu tempus venenatis, dolor elit posuere quam, quis adipiscing urna leo nec orci. Sed nec nulla auctor odio aliquet consequat. Ut nec nulla in ante ullamcorper aliquam at sed dolor. Phasellus fermentum magna in augue gravida cursus. Cras sed pretium lorem. Pellentesque eget ornare odio. Proin accumsan, massa viverra cursus pharetra, ipsum nisi lobortis velit, a malesuada dolor lorem eu neque.



\subsection{Fiche Technique}
\begin{table}[htp]
    \begin{center}
        \begin{tabular}{ | l | l |}
            \hline
            Day & Min Temp  \\ \hline
            Monday & 11C \\ \hline
            Tuesday & 9C  \\ \hline
            Wednesday & 10C  \\
            \hline
            \end{tabular}
            \caption{Fiche Technique Amundi}
    \end{center}
\end{table}