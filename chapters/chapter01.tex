% Chapter Template

\chapter{Organisme d’accueil} % Main chapter title

\label{Chapter01} % Change X to a consecutive number; for referencing this chapter elsewhere, use \ref{ChapterX}

\par
Avant d'entamer l’analyse du projet il est essentiel de commencer par une présentation générale des entités d’accueil, ceci en introduisant l'historique ainsi que les produits. Et donc par conséquent, Ce chapitre est une présentation sommaire du contexte de notre projet de fin d’études. Il comporte deux grandes parties : dans la première, nous allons présenter l’entreprise d’accueil Finaxys, puis nous allons présenter Amundi Asset Manager, le client de Finaxys chez lequel j'ai pu réaliser le projet.
Chaque partie est subdivisée en plusieurs sous-partie : nous exposeront l'historique de chaque entreprise, nous allons dévoiler les chiffres clés démontrant son succès économique. Nous allons ensuite expliquer brièvement l'organisation fonctionnelle ainsi que les différentes équipes de IT avec lequelle j'ai eu la chance d'interagir XX ainsi que sa démarche qualité pour la gestion de ses projets. Dans la seconde partie, nous présenterons notre client et développerons le cadre général du projet CDL; Carnet de Liaison en présentant brièvement ses modules, notamment les modules Adhérent et Suivi qui font l’objet de notre PFE. Dans la dernière partie, nous allons révéler la conduite de projet pour laquelle nous avons opté afin de mener efficacement notre projet.XX


\section{Finaxys}
\par
Finaxys est une société de consultants spécialisée en IT finance, banque et assurance. Elle permet aux acteurs de ce marché de bénéficier d’experts pour l’analyse, le cadrage et la définition des projets IT, comme pour leur déploiement et implémentation.

\subsection{Présentation}

Fondée en 2007, FINAXYS est une société de conseil spécialisée en IT finance, banque et assurance. Elle permet aux acteurs de ce marché de bénéficier d’experts pour l’analyse, le cadrage et la définition des projets IT, comme pour leur déploiement et implémentation. Elle compte plus de 350 collaborateurs répartis entre Paris, Londres et Bruxelles. Grâce à son personnel très actif et à la qualité de ses collaborateurs, aujourd’hui, FINAXYS a réussi à s’imposer parmi les leaders de son marché à Paris avec un chiffre d’affaire de plus de 28 Millions d’euros en 2016. 
\par Depuis janvier 2010, FINAXYS suit sa transformation des systèmes d’information et la massification des prestations IT de ses clients par la mise en place des centres de service sur-mesure. De plus, en 2011, cette dernière a pu enrichir, développer et diversifier ses offres avec l’acquisition de NOVACTOR, société de conseil spécialisé en méthodologie outsourcing et sécurité IT. 
\par Dans le but de développer ses activités à l’échelle internationale, FINAXYS a lancé une première filiale à Londres en 2010, puis une autre filiale BRAINS, début 2013, dédiée aux innovations technologiques dans les domaines de la mobilité, réseaux sociaux, cloud computing et Big Data et également un bureau à Bruxelles en janvier 2014.
\par FINAXYS a décidé de s’investir dans le domaine de la recherche et développement par l’acquisition de la start-up ScaledRisk, spécialisée dans les solutions Big Data transactionnelles et temps réel pour les institutions financières.

%-----------------------------------
%	SUBSECTION 2
%-----------------------------------

\subsection{Domaine d'activité}
FINAXYS accompagne ses clients sur l’ensemble du cycle de vie projet, du cadrage à la recette, accompagné d’une démarche qualité intégrée. FINAXYS est reconnue dans une large palette d’expertises et intervient dans nombreux domaines dont nous citons :
\begin{itemize}  
    \item Conseil SI et fonctionnel
    \item Architecture des systèmes d’information
    \item Conception et développement d’application
    \item Support fonctionnel et technique
    \item Ingénierie Financière
    \item Maitrise d’ouvrage
    \item Intégration de progiciels
    \item Ingénierie de production
\end{itemize}

\subsection{Finaxys Academy} 
\par La FINAXYS Academy, c’est un programme d'incubation interne des jeunes talents, qui forme les consultants pendant leur stage de fin d’études. C'est un passage obligatoire pour tout stagiaire afin de mener à bien ses six mois de stage. 
\par Notre promotion (Avril 2020) se compose de 8 stagiaires de différents profils. Le programme de la formation a été remodélisé afin de mieux s'adapter aux circonstances actuelles (le début du stage était programmé pour le début avril, ce qui coïncidait avec le pic de l'épidémie du COVID-19 en France). Le mois d'incubation a été condensé en trois semaines de formation, au programme un apprentissage sur les métiers et practices de FINAXYS. Parmi les sujets abordés les best practices de programmation (SOLID, xDD), des formations sur les méthodologies Agile, les design patterns ou encore des formations Big Data (Spark), Data Science (IA et ML), DevOps, stack ELK  et Blockchain. Avec en plus un projet collectif en utilisant la stack MERN (MongoDB, Express JS, React JS et Node JS) et un autre en utilisant la stack ELK (Elasticsearch, Logstash et Kibana). Une part importante également dédiée à la formation finance et RH (Préparation du dossier compétence et des entretiens clients). 
\par Nous avions ainsi accès à la plateforme de formation PluralSight. Cette plateforme offre des formations vidéos et des supports pour se perfectionner sur plusieurs technologies, langages, modèles de programmation et architecture en accès illimité pendant notre formation.
\par PluralSight représente moins de 20\% de la totalité de la formation du mois d'incubation. Le reste a été assuré par les différents ingénieurs de Finaxys notamment Lionel Beltrando pour la partie Blockchain, Nacim Kacel pour la formation SOLID, TDD et Patterns et finalement Rachid Agoudar responsable de la DT Finaxys sur toutes les autres formations (Big Data, Data science, DDD, sécurité, POO, DevOps \dots).  \\
Chaque session d'e-learning a été accompagnée d'une deuxième session avec un ingénieur pour fixer ce qui a été acquis sur PluralSight ainsi que d'un mini-rapport.

\pagebreak
\subsubsection{Programme de formation}
La plateforme Pluralsight nous a permis de nous former sur les cours suivants :
\begin{itemize}  
    \item Programmation JAVA : Java Core, Collections, Design Pattern, …
    \item Jenkins 2 : Automatisation de déploiement d’application.
    \item Ansibe \& Docker (Continuous Delivery) : Automatisation de tâches et plateforme de déploiement d’application
    \item Git Fundamentals : Gestionnaire de version de code.
    \item Maven Fundamentals : Gestion des dépendances (bibliothèques JAR).
    \item Scrum Fundamentals : Gestion de projet Agile avec la méthode Scrum.
\end{itemize}
Les ingénieurs Finaxys nous ont formé sur les cours suivants :
\begin{itemize}  
    \item Formation POO: POO Java et Design Pattern
    \item Formation Big Data: Spark, Architecture et Practice.
    \item Formation DevOps: Infra-as-code (IaC) - Projet
    \item Formation Blockchain: Généralité.
    \item Formation Blockchain: Présentation technique de la blockchain interne.
    \item Formation Data Science: Intelligence Artificielle et Machine Learning
    \item Formation stack ELK : Présentation
    \item Formation stack ELK : Projet
    \item Formation Architecture et patterns: Microservice, MVC, Monolith, Hexogonale \dots
    \item Formation Best Practices: SOLID, TDD et Patterns
    \item Formation DDD: Domain Design Driven
    \item Formation Agile : Scrum
    \item Formation Agile : Modèle Spotify
    \item Formation FrontEnd: Advanced Frontend technologies
    \item Formation Développement Applicatif: MERN stack (Projet)
\end{itemize}
Formations non technique :
\begin{itemize}  
    \item Formation RH: Produire un bon dossier de compétences.
    \item Formation RH: Préparer un entretien client.
    \item Formation RH: Coaching CV
    \item Formation RH: Préparer sa conférence (Projet: Préparer une conférence et la présenter devant l'équipe Finaxys)
    \item Formation Finance: Introduction à la finance des marchés
    \item Formation Finance: Le principe de valorisation des obligations
    \item Formation Finance: Les risques des marchés financiers
\end{itemize}

\subsection{Fiche Technique}

\def\arraystretch{1.8}
\begin{center}
    \begin{tabularx}{0.8\textwidth} { 
        | >{\raggedright\arraybackslash}X
        | >{\centering\arraybackslash}X | }
    \hline
    Nom de la société   &   Finaxys \\
    \hline
    Statut Juridique    &   Société par actions simplifiée (SAS) \\
    \hline
    Siege social        &   Tour INITIALE – 1 Terrasse Bellini,  92919 Paris la Défense  \\
    \hline
    Nombre de sites     &   3 (Paris, Londres, Bruxelles) \\
    \hline
    Date de création    &   Novembre 2007  \\
    \hline
    Secteur d’activité  &   Conseil en systèmes et logiciels informatiques  \\
    \hline
    Chiffre d’affaire   &   + 41.7 Millions d'Euros (2018)  \\
    \hline
    Clients principaux  &   Amundi Asset Management, Société Générale, BNP Paribas, HSBC \dots  \\
    \hline
    Effectif            &   350 employés\\
    \hline
    Certification       &   XXXXXXXXX  \\
    \hline
    \end{tabularx}
    \begin{table}[htp]
        \caption{Fiche Technique Finaxys}
    \end{table}
\end{center}

\section{Amundi Asset Management}

Sed ullamcorper quam eu nisl interdum at interdum enim egestas. Aliquam placerat justo sed lectus lobortis ut porta nisl porttitor. Vestibulum mi dolor, lacinia molestie gravida at, tempus vitae ligula. Donec eget quam sapien, in viverra eros. Donec pellentesque justo a massa fringilla non vestibulum metus vestibulum. Vestibulum in orci quis felis tempor lacinia. Vivamus ornare ultrices facilisis. Ut hendrerit volutpat vulputate. Morbi condimentum venenatis augue, id porta ipsum vulputate in. Curabitur luctus tempus justo. Vestibulum risus lectus, adipiscing nec condimentum quis, condimentum nec nisl. Aliquam dictum sagittis velit sed iaculis. Morbi tristique augue sit amet nulla pulvinar id facilisis ligula mollis. Nam elit libero, tincidunt ut aliquam at, molestie in quam. Aenean rhoncus vehicula hendrerit.
\subsection{Un Asset Manager}

\par Par abus de langage le terme « asset manager » est ici utilisé pour décrire le marché des biens immobiliers ; les Anglo-saxons associent le terme asset management à une gestion beaucoup plus étendue d'actifs, comprenant entre autres les obligations, les actions, les produits dérivés, les matières premières...
\par L’asset manager immobilier est un terme anglo-saxon, apparu en France dans les années 1990, qui définit le métier de gestionnaire d'actifs immobiliers d'entreprise comme un ensemble de services aux propriétaires, de l'acquisition du bien à sa cession.
\par L’asset manager immobilier est le responsable de la gestion d'un portefeuille d'actifs immobiliers pour le compte de tiers.
\par Il est le garant de la rentabilité des biens immobiliers attendue par les investisseurs, assurée par sa connaissance polyvalente des marchés de l'immobilier d'entreprise et d'habitation, sa maîtrise des baux commerciaux et du droit immobilier, ses compétences financières tant au plan de la modélisation de flux de trésorerie et scénarios immobiliers qu'il doit optimiser que du reporting aux investisseurs, sa capacité à penser et suivre des plans de travaux sur les immeubles, de la restructuration lourde au développement et son talent de négociateur tant pour la commercialisation des actifs que pour ses relations avec les différents intervenants et prestataires extérieurs.
\par Ce métier est apparu en France vers la fin des années 1990, après la crise immobilière sans précédent qu'avait connue le pays, à la suite de la spéculation de nombreux acteurs financiers anglo-saxons qui avaient su saisir les opportunités durant cette période et avaient en particulier apporté des techniques d'investissement et de gestion encore inconnues en France.
\par Une pyramide des métiers s'est créée et régit aujourd'hui le secteur de l'immobilier d'entreprise : l’investment manager immobilier (responsable d'investissements), l’asset manager immobilier (gestionnaire d'actifs immobilier représentant du propriétaire), le fund manager immobilier (responsable de fonds d'investissement immobilier), le portfolio manager immobilier (responsable des relations avec les clients et du reporting financier), le property manager (gestionnaire locatif et technique), le facility manager (responsable des services aux utilisateurs de l'immeuble).
\par Selon les sociétés, il est possible de trouver des définitions différentes de ces métiers ; l’asset manager immobilier peut ainsi, en sus de sa casquette de gestionnaire, être responsable de l'investissement en amont et de la vente de l'actif à un tiers.
\par L’asset manager immobilier exerce son métier sur des types d'actifs variés : bureaux, centres commerciaux, commerces de pied d'immeuble, logistique, entrepôts, habitation, hôtels… Il se doit de maîtriser les particularités afférentes à chacun de ces secteurs d'activités, par exemple : le marché de l'emploi et du tertiaire pour les bureaux, l'actualité des enseignes et la consommation des ménages pour le commerce, la réglementation des installations classées pour la logistique, etc.

\subsection{Historique}

Nunc posuere quam at lectus tristique eu ultrices augue venenatis. Vestibulum ante ipsum primis in faucibus orci luctus et ultrices posuere cubilia Curae; Aliquam erat volutpat. Vivamus sodales tortor eget quam adipiscing in vulputate ante ullamcorper. Sed eros ante, lacinia et sollicitudin et, aliquam sit amet augue. In hac habitasse platea dictumst.

%-----------------------------------
%	SUBSECTION 2
%-----------------------------------

\subsection{Domaine d'activité}
Morbi rutrum odio eget arcu adipiscing sodales. Aenean et purus a est pulvinar pellentesque. Cras in elit neque, quis varius elit. Phasellus fringilla, nibh eu tempus venenatis, dolor elit posuere quam, quis adipiscing urna leo nec orci. Sed nec nulla auctor odio aliquet consequat. Ut nec nulla in ante ullamcorper aliquam at sed dolor. Phasellus fermentum magna in augue gravida cursus. Cras sed pretium lorem. Pellentesque eget ornare odio. Proin accumsan, massa viverra cursus pharetra, ipsum nisi lobortis velit, a malesuada dolor lorem eu neque.



\subsection{Fiche Technique}

\begin{center}
    \begin{tabularx}{0.8\textwidth} { 
        | >{\raggedright\arraybackslash}X
        | >{\centering\arraybackslash}X | }
    \hline
    Nom de la société   &   Amundi Asset Management - Amundi AM \\
    \hline
    Statut Juridique    &   Société par actions simplifiée (SAS)  \\
    \hline
    Siege social        &   90, boulevard Pasteur - 75015 Paris  \\
    \hline
    Siege social        &   + 40 Pays (Dont Tokyo, Dublin, Boston, Londres \dots) \\
    \hline
    Date de création    &   Avril 2001  \\
    \hline
    Secteur d’activité  &   Gestion de fonds  \\
    \hline
    Chiffre d’affaire   &   + 1,17 Milliard d'Euros (2018)  \\
    \hline
    Fonds sous gestion  &   + 1600 Milliard d'Euros (2019)  \\
    \hline
    Clients             &   + 100 Millions (dont 1500 clients institutionnels)  \\
    \hline
    Effectif            &   2000 employés\\
    \hline
    Certification       &   XXXXXXXXX  \\
    \hline
    \end{tabularx}
    \begin{table}[htp]
        \caption{Fiche Technique Amundi AM}
    \end{table}
\end{center}