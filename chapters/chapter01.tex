\setlength{\parskip}{0.8em}



\chapter{Organisme d’accueil : Finaxys} % Main chapter title

\label{Chapter01} % Change X to a consecutive number; for referencing this chapter elsewhere, use \ref{ChapterX}

\par
Avant d'entamer l’analyse du projet il est essentiel de commencer par une présentation générale des entités d’accueil, ceci en introduisant l'historique, les produits ainsi que le déroulement. Et donc par conséquent, Cette première partie du rapport est une présentation sommaire du contexte de notre projet de fin d’études. Il comporte deux chapitres : dans le premier, nous allons présenter l’entreprise d’accueil Finaxys, puis nous allons présenter Amundi Asset Manager, le client de Finaxys chez lequel j'ai pu réaliser le projet.
Chaque chapitre est subdivisé en plusieurs sections : nous exposeront l'historique de chaque entreprise, nous allons dévoiler les chiffres clés démontrant son succès économique. Nous allons ensuite expliquer brièvement l'organisation fonctionnelle et le déroulement du premier mois du stage depuis les premieres formations et jusqu'à la recherche de mission. Dans la seconde partie, nous présenterons notre client et développerons le cadre général du projet MediaPlusCore ainsi que celui d'Alto Investment Research sans pour autant présenter en détail les deux projets.
\\~\\
\par
Finaxys est une société de consultants spécialisée en IT finance, banque et assurance. Elle permet aux acteurs de ce marché de bénéficier d’experts pour l’analyse, le cadrage et la définition des projets IT, comme pour leur déploiement et implémentation.

\section{Présentation}

Fondée en 2007, FINAXYS est une société de conseil spécialisée en IT finance, banque et assurance. Elle permet aux acteurs de ce marché de bénéficier d’experts pour l’analyse, le cadrage et la définition des projets IT, comme pour leur déploiement et implémentation. Elle compte plus de 350 collaborateurs répartis entre Paris, Londres et Bruxelles. Grâce à son personnel très actif et à la qualité de ses collaborateurs, aujourd’hui, FINAXYS a réussi à s’imposer parmi les leaders de son marché à Paris avec un chiffre d’affaire de plus de 28 Millions d’euros en 2016. 
\par Depuis janvier 2010, FINAXYS suit sa transformation des systèmes d’information et la massification des prestations IT de ses clients par la mise en place des centres de service sur-mesure. De plus, en 2011, cette dernière a pu enrichir, développer et diversifier ses offres avec l’acquisition de NOVACTOR, société de conseil spécialisé en méthodologie outsourcing et sécurité IT. 
\par Dans le but de développer ses activités à l’échelle internationale, FINAXYS a lancé une première filiale à Londres en 2010, puis une autre filiale BRAINS, début 2013, dédiée aux innovations technologiques dans les domaines de la mobilité, réseaux sociaux, cloud computing et Big Data et également un bureau à Bruxelles en janvier 2014.
\par FINAXYS a décidé de s’investir dans le domaine de la recherche et développement par l’acquisition de la start-up ScaledRisk, spécialisée dans les solutions Big Data transactionnelles et temps réel pour les institutions financières.

%-----------------------------------
%	section 2
%-----------------------------------

\subsection{Domaine d'activité}
FINAXYS accompagne ses clients sur l’ensemble du cycle de vie projet, du cadrage à la recette, accompagné d’une démarche qualité intégrée. FINAXYS est reconnue dans une large palette d’expertises et intervient dans nombreux domaines dont nous citons :
\begin{itemize}  
    \item Conseil SI et fonctionnel
    \item Architecture des systèmes d’information
    \item Conception et développement d’application
    \item Support fonctionnel et technique
    \item Ingénierie Financière
    \item Maitrise d’ouvrage
    \item Intégration de progiciels
    \item Ingénierie de production
\end{itemize}

\section{Finaxys Academy} 
\par La FINAXYS Academy, c’est un programme d'incubation interne des jeunes talents, qui forme les consultants pendant leur stage de fin d’études. C'est un passage obligatoire pour tout stagiaire afin de mener à bien ses six mois de stage. 
\par Notre promotion (Avril 2020) se compose de 8 stagiaires de différents profils. Le programme de la formation a été remodélisé afin de mieux s'adapter aux circonstances actuelles (le début du stage était programmé pour le début avril, ce qui coïncidait avec le pic de l'épidémie du COVID-19 en France). Le mois d'incubation a été condensé en trois semaines de formation, au programme un apprentissage sur les métiers et practices de FINAXYS. Parmi les sujets abordés les best practices de programmation (SOLID, xDD), des formations sur les méthodologies Agile, les design patterns ou encore des formations Big Data (Spark), Data Science (IA et ML), DevOps, stack ELK  et Blockchain. Avec en plus un projet collectif en utilisant la stack MERN (MongoDB, Express JS, React JS et Node JS) et un autre en utilisant la stack ELK (Elasticsearch, Logstash et Kibana). Une part importante également dédiée à la formation finance et RH (Préparation du dossier compétence et des entretiens clients). 
\par Nous avions ainsi accès à la plateforme de formation PluralSight. Cette plateforme offre des formations vidéos et des supports pour se perfectionner sur plusieurs technologies, langages, modèles de programmation et architecture en accès illimité pendant notre formation.
\par PluralSight représente moins de 20\% de la totalité de la formation du mois d'incubation. Le reste a été assuré par les différents ingénieurs de Finaxys notamment Lionel Beltrando pour la partie Blockchain, Nacim Kacel pour la formation SOLID, TDD et Patterns et finalement Rachid Agoudar responsable de la DT Finaxys sur toutes les autres formations (Big Data, Data science, DDD, sécurité, POO, DevOps \dots).  \\
Chaque session d'e-learning a été accompagnée d'une deuxième session avec un ingénieur pour fixer ce qui a été acquis sur PluralSight ainsi que d'un mini-rapport.


\subsection{Programme de formation}
La plateforme Pluralsight nous a permis de nous former sur les cours suivants :
\begin{itemize}  
    \item Programmation JAVA : Java Core, Collections, Design Pattern, …
    \item Jenkins 2 : Automatisation de déploiement d’application.
    \item Ansibe \& Docker (Continuous Delivery) : Automatisation de tâches et plateforme de déploiement d’application
    \item Git Fundamentals : Gestionnaire de version de code.
    \item Maven Fundamentals : Gestion des dépendances (bibliothèques JAR).
    \item Scrum Fundamentals : Gestion de projet Agile avec la méthode Scrum.
\end{itemize}
Les ingénieurs Finaxys nous ont formé sur les cours suivants :
\begin{itemize}  
    \item Formation POO: POO Java et Design Pattern
    \item Formation Big Data: Spark, Architecture et Practice.
    \item Formation DevOps: Infra-as-code (IaC) - Projet
    \item Formation Blockchain: Généralité.
    \item Formation Blockchain: Présentation technique de la blockchain interne.
    \item Formation Data Science: Intelligence Artificielle et Machine Learning
    \item Formation stack ELK : Présentation
    \item Formation stack ELK : Projet
    \item Formation Architecture et patterns: Microservice, MVC, Monolith, Hexogonale \dots
    \item Formation Best Practices: SOLID, TDD et Patterns
    \item Formation DDD: Domain Design Driven
    \item Formation Agile : Scrum
    \item Formation Agile : Modèle Spotify
    \item Formation FrontEnd: Advanced Frontend technologies
    \item Formation Développement Applicatif: MERN stack (Projet)
\end{itemize}
Formations non technique :
\begin{itemize}  
    \item Formation RH: Produire un bon dossier de compétences.
    \item Formation RH: Préparer un entretien client.
    \item Formation RH: Coaching CV
    \item Formation RH: Préparer sa conférence (Projet: Préparer une conférence et la présenter devant l'équipe Finaxys)
    \item Formation Finance: Introduction à la finance des marchés
    \item Formation Finance: Le principe de valorisation des obligations
    \item Formation Finance: Les risques des marchés financiers
\end{itemize}

\subsection{Recherche de mission}
\par Après avoir intégré la FINAXYS Academy, avec l'aide des ingénieurs commerciaux FINAXYS, les stagiaire se font proposer des missions qui correspondent a leur profils. En effet, même s’il était possible de passer son stage sur des projets en interne, la majorité avait pour objectif d’intégrer un projet chez un client de Finaxys.
\par Grace aux nombreuses réunions avec les responsables Ressources-Humaines, les stagiaires ont été formé sur la rédaction du CV, la formation du projet professionnel ainsi que les présentations lors des rendez-vous client. Ensuite, une rencontre avec les commerciaux pour présenter les clients ainsi que les différentes opportunités proposées chez chaque client.
\par Pour mon cas, une opportunité chez Amundi AM est apparue très tôt avant même mon début de stage, et après une rencontre avec l'ingénieur commercial FINAXYS Baptiste Moisy, la responsable Ressources-Humaines Marine Mahé ainsi que le responsable de la DT Rachid Agoudar, on a pu me former brièvement ainsi sur le déroulement de l'entretien ainsi que des éventuelles questions techniques possibles. La mission a été validé quelques jours après l'entretien client le vendredi 27 février 2020. 

\section{Fiche Technique}
\def\arraystretch{1.7}
\begin{center}
    \begin{tabularx}{0.9\textwidth} { 
        | >{\raggedright\arraybackslash}X
        | >{\centering\arraybackslash}X | }
    \hline
    Nom de la société   &   Finaxys \\
    \hline
    Statut Juridique    &   Société par actions simplifiée (SAS) \\
    \hline
    Siege social        &   Tour INITIALE – 1 Terrasse Bellini,  92919 Paris la Défense  \\
    \hline
    Nombre de sites     &   3 (Paris, Londres, Bruxelles) \\
    \hline
    Date de création    &   Novembre 2007  \\
    \hline
    Secteur d’activité  &   Conseil en systèmes et logiciels informatiques  \\
    \hline
    Chiffre d’affaire   &   + 41.7 Millions d'Euros (2018)  \\
    \hline
    Clients principaux  &   Amundi Asset Management, Société Générale, BNP Paribas, HSBC \dots  \\
    \hline
    Effectif            &   350 employés\\
    \hline
    Certification       &   XXXXXXXXX  \\
    \hline
    \end{tabularx}
    \begin{table}[htp]
        \caption{Fiche Technique Finaxys}
    \end{table}
\end{center}

\chapter{Amundi Asset Management}

Amundi est une société française de gestion d'actifs. Avec 1 653 milliards d'euros d'encours gérés à fin 2019, elle se classe à la première place des entreprises de gestion d'actifs en Europe et parmi les principaux acteurs mondiaux de ce secteur.
Créée le 1er janvier 2010, la société est issue de la fusion entre les activités de gestion d'actifs du Crédit agricole (CAAM) et de la Société générale (SGAM). Depuis novembre 2015, le groupe Amundi est coté en Bourse sur Euronext en étant majoritairement détenu par Crédit agricole S.A.
Sur le plan juridique, le groupe Amundi possède Amundi Asset Management ainsi que plusieurs filiales dans l'univers de la gestion d'actifs, notamment CPR AM et BFT IM en France. En 2016, le groupe Amundi a annoncé le rachat de Pioneer Investments, filiale de gestion d'actifs d'Unicredit, et a procédé à une fusion avec cette entité en 2017.
Les activités d'Amundi sont multiples dans le domaine de la gestion d'actifs. L'entreprise est notamment présente dans le domaine de la gestion active via une gamme d'OPCVM (gestion actions, gestion obligataire, gestion diversifiée, gestion de produits structurés et gestion de trésorerie) ainsi que dans le domaine de la gestion passive en étant un émetteur d'ETF et un gestionnaire de fonds indiciels. La société est également présente sur le segment de l'investissement en actifs réels et alternatifs (immobilier et private equity notamment). Son offre s’adresse aux investisseurs particuliers et aux investisseurs institutionnels, sous forme de produits collectifs ou de mandats d’investissement. Auprès du grand public, Amundi est principalement connu pour ses activités dans le domaine de l'épargne salariale. L'entreprise dispose également d'un pôle de recherche et d'analyse réalisant des publications régulières au sujet de la conjoncture économique mondiale et de l'évolution des marchés boursiers.
Le groupe Amundi possède des bureaux dans plusieurs pays du monde, notamment en Europe, en Asie et aux États-Unis, et estime compter environ 100 millions de clients particuliers directs ou indirects ainsi que 1 500 clients institutionnels dans le monde.

\section{Asset Management}

\par L'Asset Management, en français, "Gestion d'Actifs" est sans aucun doute la seule partie des marchés financiers connue de tous le monde. Peut-être pas le terme Asset Management en lui-même, mais à coup sûr les produits qui en émanent.
\par Un Asset Manager est en effet la société, le plus souvent filiale d'une banque ou d'un assureur (le statut de filiale est obligatoire depuis 1999), qui crée et gère au quotidien les produits de placements dont chacun de nous voit la publicité sur les devantures de toutes les agences de banques et d'assurances : les OPCVM, organismes de placement collectif en valeurs mobilières (SICAV et FCP), ou plus généralement les produits d'épargne collective.
\par Plus simplement, l'Asset Manager est la société à qui les particuliers et les entreprises peuvent confier de l'argent pour qu'il soit géré au travers d'un fonds, qui est le véhicule de placement. Les Asset Managers font donc de la « gestion pour compte de tiers ».
\par Ainsi, les principaux clients des sociétés de gestion sont :
\begin{itemize}
    \item Des entreprises souhaitant placer leurs excédents de trésorerie
    \item Des Caisses de Retraites
    \item Des institutions financières investissant pour compte propre ou distribuant les OPCVM à leurs clients (Retail, banque privée, banque de grande clientèle…) 
\end{itemize}

\par Les clients d'un Asset Manager sont donc sensiblement les mêmes (hors Retail) que ceux d'une BFI. Alors que la gestion d'actif adressera plutôt une offre d'épargne packagée, souvent à moyen ou long terme, les BFI axeront plutôt leur offre sur du conseil lors d'opérations de haut de bilan, plutôt à court terme.

\section{Présentation}
\subsection{Historique}
\subsubsection{Création}

\par Fin 2008, Crédit agricole et Société générale décident de fusionner au sein d'une nouvelle société leur filiale respective de gestion d'actifs, CAAM et SGAM. Les deux filiales gèrent alors chacune une gamme d'OPCVM constituée de fonds obligataires, de fonds actions, de fonds alternatifs et de produits structurés ainsi qu'une gamme d'ETF.
\par Un accord préliminaire est signé le 26 janvier 2009 entre les deux parties prenantes, puis un accord définitif est signé le 9 juillet 2009, stipulant que Crédit agricole possèdera 75\% de la nouvelle entreprise créée et Société générale 25\%, avec une direction générale assurée par Yves Perrier, alors directeur général de CAAM. Pendant la mise en place du projet, la future société est temporairement désignée sous les noms génériques de « CAAM SGAM » et de « Newco » avant que le nom « Amundi » soit officiellement annoncé le 23 octobre 2009. La société est créée le 1er janvier 2010 suite à l'accord de la Commission européenne pour procéder à la fusion. La fusion effective entre les équipes a lieu de manière progressive dans le courant de l'année 2010 et s'accompagne de 260 suppressions de postes dans le monde dont 185 en France, et la création d'environ 60 nouveaux postes dans les branches de gestion des risques et de distribution commerciale.
\par Avec 670 milliards d'euros d'encours sous gestion à la veille de sa création, Amundi apparaît alors comme le troisième plus grand acteur de la gestion d'actifs en Europe derrière Axa et Allianz, et fait partie des 10 plus grandes sociétés de gestion d'actifs dans le monde.

\subsubsection{Développement et Expansion}

\par Les fonds d'Amundi sont à l'origine principalement distribués par les réseaux bancaires de ses deux actionnaires majoritaires : Crédit agricole, LCL (filiale du Crédit agricole), Société générale et Crédit du Nord (filiale de Société générale), qui totalisent à eux seuls plus de 70\% des flux de collecte nette d'Amundi à ses débuts, le solde étant constitué d'investisseurs institutionnels. La société de gestion élargit progressivement sa base d'investisseurs : en 2015, plus de 60\% des encours de la société proviennent de sources extérieures à ces réseaux bancaires.
\par En 2012, Amundi élabore un accord de distribution avec la société de gestion TOBAM et prend à cette occasion une participation de 10,6\% au sein du capital de la société (participation qui montera à 20\% en 2016).
\par En juin 2013, Amundi annonce l'acquisition de la société Smith Breeden Associates aux États-Unis, qui devient effective en octobre 2013. La société, spécialisée dans le domaine de la gestion obligataire en dollars, gère alors 6,4 milliards de dollars d'encours, soit 4,9 milliards d'euros. À l'issue de l'opération, Smith Breeden Associates est renommée « Amundi Smith Breeden LLC » et devient le siège d'Amundi pour ses activités en Amérique du Nord.
\par En octobre 2014, Amundi rachète 100\% du capital de Bawag PSK Invest, filiale de gestion de la banque autrichienne Bawag PSK, marquant l'arrivée d'Amundi sur le marché autrichien. Bawag PSK Invest, qui devient une franchise d'Amundi, gère alors 4,6 milliards d'euros d'encours à travers une gamme de 78 fonds. Le rachat comprend un accord avec la banque Bawag PSK pour que celle-ci distribue les fonds d'Amundi à travers son réseau d'environ 500 agences en Autriche. Peu après, Amundi annonce le rachat de 87,5\% du capital de Kleinwort Benson Investors, société de gestion basée à Dublin disposant d'antennes à Boston et New York et gérant 7,6 milliards d'euros d'encours.
\par Dans le courant de l'année 2014, Crédit agricole S.A. renforce sa participation au capital d'Amundi en rachetant à Société générale 5\% du capital de la société pour 337,5 millions d'euros. Crédit agricole S.A. contrôle dès lors 80\% du capital d'Amundi.

\subsubsection{Initial public offering (IPO)}
\par Le 17 juin 2015, Crédit agricole et Société générale annoncent leur intention d'introduire Amundi en bourse avant la fin de l'année. Les Echos estiment alors que le groupe devrait être valorisé entre 7 et 10 milliards d'euros en bourse, une taille « aux dimensions du CAC 40 » selon le journal. L'introduction en bourse est annoncée comme une occasion pour Société générale de céder les 20\% du capital qu'elle détient dans l'entreprise.
\par Le groupe Amundi est introduit en bourse sur Euronext le 12 novembre 2015. Société générale cède comme prévu l'entièreté de sa participation sur le marché et Crédit agricole cède 2\% du capital d'Amundi à la Banque agricole de Chine, tout en prévoyant de céder d'autres parts ultérieurement en conservant 75\% du capital d'Amundi. Lors de son entrée en bourse, le groupe Amundi dispose ainsi de 20\% de flottant et sa capitalisation boursière est de 7,5 milliards d'euros.
\par L'action du groupe est introduite sur le marché à 45 euros et termine sa première séance au-dessus de 47 euros. La presse spécialisée met alors en contraste la réussite de l'opération avec l'annulation des introductions en bourse de Deezer et d'Oberthur Technologies survenues précédemment dans un contexte de marché rendu difficile par les turbulences boursières de l'été 2015.

\subsubsection{Pioneer Investments}
\par En décembre 2016, Amundi annonce le rachat à 100\% de Pioneer Investments, filiale de gestion d'actifs de la banque italienne Unicredit. Le rachat de Pioneer Investments est annoncé dans la presse comme la plus importante opération de fusion-acquisition au sein du groupe Crédit agricole depuis de nombreuses années. Le rachat de Pioneer Investments est effectué pour un montant de 3,5 milliards d'euros. La transaction est financée par Amundi à hauteur de 1,5 milliard d'euros, via une émission de dette pour 600 millions d'euros et par une opération d'augmentation de capital de 1,4 milliard d'euros garantie par Crédit agricole. Fin 2017, au terme de l'opération, Crédit agricole ne possède plus que 70\% du capital d'Amundi, contre 75\% précédemment.
\par Finalisée le 3 juillet 2017, l'opération permet à Amundi d'intégrer 242,9 milliards d'euros à ses encours sous gestion au troisième trimestre 2017, passant de 1 121 milliards d'euros sous gestion fin juin 2017 à 1 400 milliards fin septembre. Amundi devient ainsi le huitième plus grand acteur mondial de la gestion d'actifs. Fin 2017, Amundi gère 1 426 milliards d'euros d'encours.
\par Le rachat permet notamment à Amundi d'agrandir son réseau de distribution en Italie, en Allemagne et en Autriche, où Pioneer Investments était déjà implantée, tout en développant ses expertises de gestion. L'Italie devient ainsi le second marché d'Amundi après la France. Aux États-Unis, le nom de Pioneer est conservé en étant associé à celui d'Amundi pour créer la marque Amundi Pioneer.
\par À l'annonce du rachat, Amundi précise envisager un plan de réduction de ses effectifs de 450 personnes sur un total de 5 000 employés dans le monde. En France, un plan de suppression de 134 postes (sur un total de 2 000 salariés en France) est annoncé en octobre 2017.

%-----------------------------------
%	SUBSECTION 2
%-----------------------------------

\subsection{Domaine d'activité}

\subsubsection{Asset Management}
\par L'activité d'Amundi consiste à gérer des fonds d'investissement au sein desquels des investisseurs particuliers, des investisseurs institutionnels ou des entreprises peuvent respectivement placer, collectivement ou de façon individuelle (via mandats et fonds dédiés), leur épargne, leurs capitaux et leur trésorerie, en déléguant à Amundi la gestion de cet argent. Le cœur de métier d'Amundi est donc la « gestion pour compte de tiers » et le chiffre d'affaires du groupe (produit net bancaire) est constitué des frais prélevés sur les encours dont Amundi assure la gestion (frais de souscription et frais de gestion annuels principalement). Amundi gère différents types de fonds, avec notamment une gamme d'OPCVM et une gamme d'ETF, auxquels s'ajoutent des fonds investis en actifs réels et alternatifs, notamment en immobilier, ainsi qu'une offre de produits structurés.
\subsubsection{OPCVM}
Amundi gère une large gamme d'OPCVM qui se scinde en deux branches principales : l'investissement en actions (fonds actions) et l'investissement en obligations (fonds obligataires et fonds monétaires). À ces deux branches s'ajoutent deux pôles transversaux avec la « gestion diversifiée », ayant pour but de mêler au sein d'un même fonds différentes classes d'actifs (actions, obligations, actifs monétaires, immobilier) et la gestion dite de « performance absolue », qui cherche à dégager chaque année des rendements supérieurs ou égaux à la performance générée par des placements monétaires. La gestion OPCVM était à l'origine la principale expertise de CAAM et de SGAM, dont la collecte était notamment assurée par les réseaux bancaires du Crédit agricole et de Société générale, les conseillers bancaires proposant généralement à leurs clients particuliers et professionnels d'investir dans des OPCVM de leur maison-mère. Depuis la création d'Amundi, la gamme d'OPCVM du groupe est toujours mise en avant par les conseillers bancaires des réseaux du Crédit agricole et de Société générale, mais également de manière grandissante auprès d'investisseurs particuliers ou professionnels hors réseau bancaire, en France et à l'international. Auprès du grand public, Amundi est principalement connue comme étant une société gérant des fonds d'épargne salariale pouvant être souscrits dans le cadre d'un plan d'épargne d'entreprise. Amundi est en effet le leader de l'épargne salariale en France en gérant 42\% des encours totaux sur ce type de support33 et en proposant depuis 2017 un outil de robo-advisoring permettant d'aider les épargnants à choisir la répartition de leur épargne salariale en fonction de leur appétence pour le risque.
\subsubsection{ETF}
\par Amundi est également présente dans le domaine de la gestion passive en étant un émetteur d'ETF commercialisés sous la marque « Amundi ETF ». À fin 2017, le groupe gère 38 milliards d'euros sur ce type de produits. Historiquement, l'activité d'Amundi ETF est héritée de CASAM, branche de CAAM gérant 65 ETF fin 2009 avant la création d'Amundi, la gamme d'ETF de SGAM ayant pour sa part été transférée en 2009 à Lyxor AM, filiale de Société générale. Amundi ETF recouvre une gamme de produits divisée en deux catégories : les ETF actions (répliquant notamment la performance d'indices boursiers nationaux, régionaux ou sectoriels) et les ETF obligataires (répliquant la performance des obligations d'État ou d'entreprises).
\subsubsection{Actifs réels et alternatifs}
\par Depuis septembre 2016, Amundi réunit l'ensemble de ses investissements en immobilier, en dette privée, en private equity et en infrastructures sur une plateforme unique dédiée aux « actifs réels et alternatifs ». Lors de sa création, cette branche regroupe 34 milliards d'euros d'actifs sous gestion avec pour objectif d'atteindre 70 milliards d'euros d'actifs sous gestion à l'horizon 2020. Ce pôle est principalement dominé par les investissements en immobilier, qui représentaient à son lancement 14 milliards d'euros d'actifs sous gestion. Parmi les bâtiments acquis par Amundi depuis le lancement de cette plate-forme se trouvent notamment des immeubles du quartier d'affaires de La Défense, dont Cœur Défense (acquis en 2017 en partenariat avec Crédit agricole assurances et Primonial REIM) ainsi que la Tour Hekla (acquise en VEFA en 2017 en partenariat avec Primonial REIM). Les biens immobiliers possédés par Amundi sont portés par des OPCI et des SCPI. Dans le domaine de l'investissement en private equity, Amundi a créé « Amundi Private Equity Funds », une filiale destinée à acquérir des participations dans des entreprises non cotées. L'investissement en dette privée recouvre pour sa part le financement de dettes et de stocks d'entreprises, la presse financière ayant notamment relayé en 2017 le lancement d'un fonds de dette adossé à des stocks physiques de jambon et de parmesan en Italie. Dans le domaine des investissements en infrastructures, Amundi a développé un partenariat avec EDF en créant « Amundi Transition Energétique », filiale détenue à 60\% par Amundi et 40\% par EDF, visant à financer des projets liés aux énergies renouvelables et à l'efficacité énergétique. Amundi a également développé en 2017 un partenariat avec le CEA en créant « Supernova Invest », une société détenue à 40\% par Amundi Private Equity visant à investir dans des projets d'innovations technologiques sur le territoire français.
\subsubsection{Investissement socialement responsable}
\par En 2018, Amundi prend de nouvelles initiatives dans le domaine de l'ISR. En mars 2018, dans l'univers obligataire (gestion OPCVM), Amundi s’associe à la Société financière internationale (membre du Groupe de la Banque mondiale) pour lancer un fonds d'« obligations vertes » (green bonds) émises par des pays émergents. Avec 1,42 milliard de dollars d'encours à son lancement, le véhicule est alors le plus grand fonds de green bonds à l'échelle mondiale. En octobre 2018, Amundi annonce par ailleurs son « Plan d'action 2021 », qui vise à généraliser à l'ensemble de ses processus d'investissement la prise en compte de critères environnementaux, sociaux et de gouvernance (critères ESG). Le principal objectif du groupe est ainsi qu'en 2021, 100\% de ses encours sous gestion soient investis en respectant des critères ESG, contre 5\% (32 milliards d'euros) lors de son lancement en 2010 et 19\% (280 milliards d'euros) lors de l'annonce du plan en 2018. Amundi s'engage également dans ce plan à tenir compte de critères ESG dans sa politique de vote au sein des assemblées générales des entreprises dont elle est actionnaire.
\subsubsection{Recherche et analyse financière}
\par Parallèlement à ses activités de gestion, Amundi s'appuie sur un département de recherche et d'analyse spécialisé dans les domaines des marchés financiers et de l'étude de la conjoncture économique mondiale. Le groupe publie gratuitement et en libre accès ses travaux de recherche économique en français et en anglais sur le site Internet « Amundi Research Center ». Amundi diffuse notamment sur ce site une publication mensuelle intitulée « Cross Asset Investment Strategy » dont des extraits sont régulièrement repris par la presse financière. Toujours dans le domaine de la recherche et de l'analyse financière, Amundi organise chaque année l'« Amundi World Investment Forum » au cours duquel différents intervenants extérieurs débattent de sujets macroéconomiques.
\par Amundi a également recours à des analyses financières externes pour accompagner ses activités de gestion. Dans le cadre de l'entrée en vigueur de la directive européenne MIF2 (MIFID) en janvier 2018, imposant notamment aux sociétés de gestion une plus grande transparence auprès de leurs clients sur ce type de coûts, Amundi a choisi de prendre en charge ces frais de recherche externe sans les facturer à ses clients.
\subsubsection{Services}
\par Amundi a également lancé en 2016 « Amundi Services », une branche visant à offrir des services à d’autres sociétés de gestion d’actifs. L’offre est inspirée de la plate-forme « Aladdin » de la société américaine BlackRock, et vise à s’adresser à des sociétés de petite et de moyenne taille souhaitant externaliser une partie de leurs activités, en particulier l’exécution des ordres, la gestion des positions, la vérification des règles d’investissement, le calcul d’indicateurs de risques, le suivi des performances et les travaux de « reportings » réglementaires. Amundi Services a pour objectif de générer 50 à 80 millions d'euros de chiffre d’affaires à l’horizon 2020, soit environ 5\% du produit net bancaire du groupe. En 2018, Amundi a annoncé un partenariat entre Goldman Sachs et sa branche Amundi Services, cette dernière assurant la gestion, le contrôle et la supervision de produits de Goldman Sachs Fund Solutions dans le domaine de la gestion quantitative et alternative.

\subsection{Amundi a l'international}
\par À l'international, Amundi possède six plates-formes de gestion à Paris, Tokyo, Londres, Milan, Boston et Dublin.
\par La société possède des bureaux dans la plupart des pays d'Europe pour assurer la distribution locale de ses produits d'investissement. Le groupe dispose notamment de filiales au Luxembourg (Amundi Luxembourg), en Suisse (Amundi Suisse, présent à Genève et Zurich), en Allemagne (Amundi Deutschland GmbH, présent à Francfort et Munich), en Autriche (Amundi Austria GmbH), en Italie (Amundi SGR SpA) et en Espagne (Amundi Iberia SGIIC). Au Luxembourg, Amundi détient également 50,04\% du capital de Fund Channel, une société de distribution de fonds d'investissement détenue en joint-venture avec BNP Paribas qui possède pour sa part 49,96\% de la société.
\par En Asie-Pacifique, qui représente son principal marché hors Europe, Amundi possède notamment une filiale au Japon (Amundi Japan) issue du rachat de Resona AM par SGAM en 2004 et dispose également de filiales à Hong Kong (Amundi Hong Kong Limited), Taïwan (Amundi Taiwan Limited), Singapour (Amundi Singapore Limited), en Thaïlande (Amundi Mutual Fund Brokerage Securities (Thailand) Company Limited), en Malaisie (Amundi Malaysia Sdn. Bhd) et en Australie (Amundi Asset Management Australia Ltd.). Amundi possède par ailleurs trois autres implantations sur le continent asiatique sous forme de joint-ventures avec des acteurs financiers locaux : en Chine, Amundi détient 33,3\% d'ABC-CA Fund Management, société de gestion issue du partenariat entre l'Agricultural Bank of China et le Crédit agricole79, en Inde, le groupe possède 37\% de SBI Mutual Fund, société de gestion détenue en partenariat avec la State Bank of India (partenariat hérité de SGAM), et en Corée du Sud, Amundi possède 30\% de NH-Amundi Asset Management, une société de gestion d'actifs détenue à 70\% par NongHyup Financial Group of Korea. En 2016, les encours cumulés de ces trois sociétés atteignaient 130 milliards d'euros, comptabilisés à 100\% dans le bilan d'Amundi.
\par Sur le continent américain, outre son implantation à Durham (Caroline du Nord) issue du rachat de Smith Breeden Associates, Amundi possède aux États-Unis un siège à Boston dédié à sa filiale Amundi Pioneer, depuis lequel sont pilotées les activités d'Amundi en Amérique du Nord13. Le groupe est également présent à Montréal (Amundi Canada Inc.). En Amérique latine, Amundi dispose notamment d'une antenne au Mexique et au Chili, toutes deux rattachées à Amundi Iberia SGIIC.
\\~\\~\\ \par Au Maghreb et au Moyen-Orient, Amundi possède principalement une antenne au Maroc (Amundi Investment Maroc) dédiée à l'investissement en immobilier. Dans le même pays, le groupe détient également 34\% de la société Wafa Gestion. Le groupe dispose enfin d'une antenne en Arménie (Amundi ACBA) et de deux antennes aux Émirats arabes unis à Abou Dhabi et à Dubaï.
\section{Fiche Technique}

\begin{center}
    \begin{tabularx}{0.8\textwidth} { 
        | >{\raggedright\arraybackslash}X
        | >{\centering\arraybackslash}X | }
    \hline
    Nom de la société   &   Amundi Asset Management - Amundi AM \\
    \hline
    Statut Juridique    &   Société par actions simplifiée (SAS)  \\
    \hline
    Siege social        &   90, boulevard Pasteur - 75015 Paris  \\
    \hline
    Siege social        &   + 40 Pays (Dont Tokyo, Dublin, Boston, Londres \dots) \\
    \hline
    Date de création    &   Janvier 2010  \\
    \hline
    Secteur d’activité  &   Gestion de fonds  \\
    \hline
    Chiffre d’affaire   &   + 1,17 Milliard d'Euros (2018)  \\
    \hline
    Fonds sous gestion  &   + 1600 Milliard d'Euros (2019)  \\
    \hline
    Clients             &   + 100 Millions (dont 1500 clients institutionnels)  \\
    \hline
    Effectif            &   2000 employés\\
    \hline
    \end{tabularx}
    \begin{table}[htp]
        \caption{Fiche Technique Amundi AM}
    \end{table}
\end{center}