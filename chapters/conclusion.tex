\par Ma mission chez Finaxys et son client Amundi fut la parfaite conclusion de mes années d’Étude à l'École nationale des sciences appliquées d'Oujda ainsi que mon semestre d'échange à l’Université Sorbonne Paris Nord. Lors de ces six mois, j’ai pu mettre en pratique les connaissances théoriques que j’ai acquises durant ma formation universitaire et par la même occasion les approfondir de manière conséquente.
\par C’est ainsi que j’ai pu découvrir le monde passionnant qu’est la gestion d’actifs, en particulier toutes les opérations qui se déroulent en « arrière-plan » lorsqu'un gérant passe un ordre en bourse, cherche une information ou même essaye de visualiser un paramètre pour prendre une decision. Véritable révélation pour moi, ce domaine est celui dans lequel je souhaite travailler pendant ma future vie professionnel.
\par L’objectif de ce stage était de découvrir les différentes missions que l’équipe Master Data Management peut rencontrer au quotidien. La réussite de mes missions est donc passée par l’apprentissage des nombreux logiciels utilisés dans mon service. A l’issue de ces 6 mois je suis donc capable de réaliser des configurations sur MediaPlus, de paramétrer des API, mettre à jour des champs DECOLOG ou encore aider les ingénieurs métier Amundi AM a prendre des décisions/actions sur Alto Investment Research. 
\par J’ai pu apporter à l’équipe un gain de productivité grâce à l’automatisation des demandes des gérants par mail, mais aussi grâce à l’organisation de l'Interface Alto Investment Research pour laquelle j’ai aussi élaboré un document indiquant les démarches à suivre ainsi qu'une documentation mapping des champs présent dans les applications MediaPlus-Core ainsi qu'Alto Investment Research.
\par Ce stage a donc été en tout point de vue très enrichissant et valorisant. Chacune des missions qui m’ont été confiée a été pour moi un réel plaisir et une vraie source de motivation. Mon travail a constitué le lien entre le métier Amundi AM et le service Amundi-ITS afin qu’ils mettent en service nos réalisations. Rencontrer les MOA, comprendre les besoins métiers et assurer le bon fonctionnement de la tâche une fois réalisée m’a permis d’avoir une bonne vision du fonctionnement global de la gestion d’actifs.
\par Les difficultés que j'ai pu rencontrer ont été pour la plupart en début du stage. Comprendre le jargon interne à Amundi comme par exemple MEP pour mise en production, equity, assets\dots a nécessité un temps d'adaptation. Appréhender tous les termes financiers en venant d’une école d’ingénieur généraliste plutôt qu'une formation spécialisée en finance m'a demandé un travail supplémentaire, mais c'est là la force même de ma formation d'ingénieur, des bases dans tous les domaines que je pourrai approfondir sans difficultés durant ma carrière professionnelle. 
\par La situation sanitaire actuelle a aussi compliqué la tache. En effet, mon début de stage se coïncidait avec les première semaines du confinement ce qui compliquait plusieurs taches, notamment la mise en place de l'environnement de travail. J’ai également pu prendre conscience de la variabilité de la quantité de travail à réaliser, certaines journées sont calmes alors que d’autres sont surchargées.
\par Je garde donc de ce stage un excellent souvenir, il constitue désormais une expérience professionnelle encourageante pour mon avenir.